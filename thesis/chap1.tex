\chapter{Mars Express, MARSIS and ionograms}

\section{Mars Express}
First of all, let us briefly introduce the spacecraft carrying all the equipment needed to acquire ionograms. Its name is \textit{Mars~Express} (MEX) and it was launched by the \textit{European Space Agency} (ESA) on 2~June~2003.

MEX arrived to Mars at its orbit with periapsis \n[km]{250} and apoapsis over \n[km]{11000} on~25~December~2003~\citep{Chicarro2004} with seven~onboard scientific instruments and a~landing module called Beagle~2. We're going to take a look at all of them in the following subsections; just Beagle~2 description is going to be rather short, because the landing sequence failed (for an unknown reason) and the lander didn't establish connection after it landed (if it landed at all)\citep[p.~4]{Chicarro2004}. 

The mission of MEX has several goals like ``global studies of the surface, subsurface and atmosphere at unprecedented spatial and spectral resolutions'' \citep[p.~viii]{Chicarro2004}. One of the goals, however, stands out among all the others. It is the search for water (or its traces) on martian surface or subsurface.

\begin{figure}
	\centering
	\includegraphics[width=140mm]{images/Mars.jpg}
	\caption{Mars Express spacecraft. Credit: ESA \citep{ESA2010}}
	\label{fig:mex}
\end{figure}

Why water? There is lots of geological evidence of former water occurrence. But before the~MEX~mission nobody had proved or refuted presence of water on~Mars in the present. Knowing more about water on Mars and its history, the scientists could postulate better hypotheses about the possibility of (former) life on the~planet \citep[p.~ix]{Chicarro2004}. \pdfcomment{Tady si nejsem uplne jisty tim, jak uvadet citace, protoze prakticky cela podkapitola cerpa z jedine knizky}

The original mission lifetime of MEX was projected up~to the~end of~2005 (which would be 1~Martian~year = 687~Earth~days) \citep{ESA2004}. However, overcoming some small problems (as the Solid State Mass Memory anomalies described in~\citep{ESA2011} or the~MARSIS~antennas deployment problems in~2004~\citep{ESA2004a,ESA2005}), MEX has worked on its science goals up to this day and its science mission was extended until~2014~\citep{ESA2013} (after 3~preceding similar extensions). Fred Jansen, MEX mission manager, said MEX had enough fuel for another 14~years of~operation (at the~beginning of~2012)~\citep{Clark2012}. So there is a hopeful prospect of further and even deeper Mars exploration (eg. \citep{Gurnett2005}~discovered an~unexpected way of~using the MARSIS~instrument so that they ``added magnetometer functionality'' to~MARSIS).

In the~next subsections you can find~out more about particular MEX instruments. The descriptions are based on~\citep{Chicarro2004} which you can see for~more detailed information.

\subsection{HRSC (\textit{High-Resolution Stereo Camera})}

\begin{figure}
	\centering
	\includegraphics[width=140mm]{images/Topographical_view_of_Amenthes_Planum.jpg}
	\caption{Example image taken by HRSC. Credit: ESA/DLR/FU Berlin (G. Neukum) \citep{Neukum2013}}
	\label{fig:hrsc_example}
\end{figure}

HRSC is a~high-resolution pushbroom\footnote{A camera that scans the image by rows perpendicular to the flight direction. See \url{http://earthobservatory.nasa.gov/Features/EO1/eo1\_2.php} for more details.} camera for~surface imaging. Its goals are to~characterize surface structure and morphology at~resolution \n[m.px^{-1}]{10} (regions of interest at \n[m.px^{-1}]{2}), surface topology at high vertical resolution, atmospheric phenomena, physical properties of the~surface and to~classify terrain and to~refine the~martian cartographic base. It is also intended to observe martian moons Phobos and Deimos during their approaches.

HRSC is~able to capture the surface at~resolution up~to~\n[m.px^{-1}]{10} with field of~view~\n[\degree]{11.9}, covering a \n[km]{52.2} wide strip of surface at~height \n[km]{250} (which is the~periapsis of~MEX). The camera consists~of 9~CCD~sensors allowing it to acquire triple stereo images in 4~colors and 5~phase angles. What is a very useful property of these images, is that they are taken nearly simultaneously and thus having the same illumination and other observational conditions (which further helps in photogrammetric processing of the images).  

HRSC also contains a~super-high-resolution camera called~SRC (\textit{Super-Resolution Channel}) aimed at~targeted observations of~particular surface details. With image resolution \n[m.px^{-1}]{2.3} and field of view \n[\degree]{0.54} it provides a detailed view of a \n{2.3}x\n[km]{2.35} large surface. Its main purpose is to~take details of~places of~interest, eg. future landing sites for other landing modules.

% achievements
Up to November~2011 HRSC had covered about~\n[\%]{88} of the martian surface \citep[pp.~72--73]{ESA2011a} and still continues to gather new data. The scientific results of HRSC are for example better exploration of fluviatile valleys \citep{Mangold2008}, dicovery of numerous glacial landforms, investigating lava flows, dicovery of ``dust devils'' (fast moving dust storms) or providing data to derive a detailed topographic model of more than \n[\%]{20} of Phobos \citep[pp.~945--949]{Jaumann2007}.

\subsection{OMEGA (\textit{Observatoire pour la Min�ralogie, l'Eau, les Glaces et l'Activit�})}
OMEGA is a medium- and high-resolution spectrometer operating in~visible and near-IR spectra (\n{0.38}--\n[{\upmu}m]{5.1} wavelength). Its medium-resolution operating mode (from heights of~\n{1500} to~\n[km]{4000}) can measure with the resolution~2--\n[km]{5} targeting at global surface coverage, while the high-resolution~mode (from the~close vicinity of~periapsis) brings resolution \n[m]{350} or~better, but will cover only a small fraction of the surface. 

As stated in~\citep[pp.~38--39]{Chicarro2004}, the main goals are to~study the~evolution of~Mars, to~detect minerals hidden to~lower resolutions, to~map mineralogical boundaries between geological units, to~reveal gradients in hydration minerals related to~fossil water flows and to~monitor features associated with~wind transportation. In~particular, it is intended to find carbonates (not found on martian surface until the launch of MEX) and water ice. It is also able to measure atmospheric pressure, CO and H$_2$O column densities and surface temperature.

Recent contributions of the OMEGA payload are e.g. confirmation of liquid water on the surface when the planet was young \citep{Loizeau2012}, discovery of infrared and ultraviolet glows in the atmosphere \citep{Bertaux2012}, proving that Mars had a hot and wet period \citep{Chevrier2007} (implying there were lots of greenhouse gases and a strong magnetic field, too \citep[p.~90]{Fletcher2009}), analyzing the south polar cap and finding out it is formed mainly of water ice \citep{Doute2007}, observation of CO$_2$ ice clouds \citep{Montmessin2007} or finding ferric oxides near the equator \citep{Masse2008}.

\subsection{MARSIS (\textit{Mars Advanced Radar for Subsurface and Ionosphere Sounding})}
\label{ssec:marisIntro}
MARSIS is a long-wavelength radar using coherent wide-band pulses for sounding of the surface, subsurface and ionospehere of Mars. For these purposes it~uses a \n[m]{40} dipole antenna (for both transmitting and receiving) and a~shorter \n[m]{7} monopole antenna (only for~receiving). Due to the used sounding frequencies ranging from \n[kHz]{100} to \n[MHz]{5.5} it is able to reach the depth about 5--\n[km]{8} under the~surface.

The primary goal of MARSIS is to detect liquid and solid water in the upper crust of Mars. There are also other objectives: subsurface geologic probing (to make a~3D characterization of the subsurface structures), surface characterization (to measure surface roughness, reflectance to radar signals and to estimate topography) and ionosphere sounding (to measure interaction between solar wind and the ionospehere) \citep[p.~51]{Chicarro2004}.

To name some results of the MARSIS instrument, we can mention revealing the layered subsurface structure of both polar caps (strongly suggesting there were oceans in distant history at these places) \citep[pp.~98--102]{Fletcher2009} along with estimating the volume of subsurface water ice in the~polar cap \citep{Phillips2008}, discovery of \textit{Medusae Fossae Formations} (the youngest surface deposits) \citep[pp.~102--105]{Fletcher2009} or mapping the ionosphere and verifying the ionospheric density models \citep[pp.~105-110]{Fletcher2009}.

One surprising and unexpected utilization of the MARSIS instrument is given by the electron cyclotron echoes found in ionograms (see section \ref{ssec:cyclotronEchoes}). It was found that they often correspond to the strength of the magnetic field, effectively allowing to measure that field and compare it to its model. Another type of echoes, the oblique ionospheric echoes (see section \ref{ssec:oblique}) were identified to correspond to the crustal magnetic field. Both these contributions were made by \citep{Gurnett2005}. 

\subsection{PFS (\textit{Planetary Fourier Spectrometer})}
PFS is IR-spectrometer (based on double-pendulum interferometer) operating in the range \n{1.2}--\n[{\upmu}m]{42} divided into two channels -- the \textit{Short Wavelength} (SW) channel (\n{1.2}--\n[{\upmu}m]{5}) and the \textit{Long Wavelength} (LW) channel (5--\n[{\upmu}m]{42}). Its spatial resolution is \n[km]{10} for SW and \n[km]{20} for LW (from altitude \n[km]{300}). PFS uses an~onboard \textit{Fast Fourier Transform} circuit to select only the data scientists are interested in.

The objectives of this device are atmospheric studies like atmospheric composition (as it can detect eg. H$_2$O, CO and CO$_2$ spectra), solid-phase surface components detection and atmospheric dust measurements. PFS also captures the vertical temperature--pressure profiles and dust and ice opacity \citep[pp.~115--116]{Chicarro2004}.

The contributions made using PFS so far are for example measuring the atmospheric temperature (finding out that there is a rather complicated situation around the peak of Olympus Mons), measuring the surface temperature, counting the atmospheric dust content, observing temperature inversion effects, detecting methane in the atmosphere (which could imply either organic life or volcanic activity, which are both unexpected phenomena), proving that the south polar cap is made mainly from CO$_2$~ice, or capturing the solar spectrum from the surroundings of Mars (which cannot be done from Earth) \citep[pp.~122--135]{Chicarro2004}.

\subsection{SPICAM (\textit{SPectroscopy for the Investigation of the Characteristics of the Atmosphere of Mars})}
The SPICAM instrument is made of two spectrometers, one operating in the UV spectrum (118--\n[nm]{320}) and the other in the near-IR (\n{1.0}--\n[{\upmu}m]{1.7}).

Many tasks have been assigned to SPICAM, the major of them being investigating ozone, H$_2$O and aerosols vertical profiles in the atmosphere. These should help constructing meteorological and dynamical atmospheric models, understanding the water vapour atmospheric cycles, characterize processes of water escape from the atmosphere, investigating the interactions between surface and atmosphere and revealing impact of aerosols on martian climate \citep[pp.~97--100]{Chicarro2004}.

One of the latest surprises brought by SPICAM is martian atmosphere is supersaturated with water vapour which further prepares conditions for water escape from the atmosphere \citep{Maltagliati2011}. Another unexpected result are nocturnal aurorae observed in the upper atmosphere, along with the (expected) NO recombination nightglow \citep{Bertaux2005}. Other results involve retrieving global spatial and temporal climatology of ozone \citep{Perrier2006}, south polar cap observations \citep[pp.~158--159]{Fletcher2009}, studies of UV dayglow \citep[pp.~160--162]{Fletcher2009}, constructing the aerosol vertical profiles \citep[pp.~175--180]{Fletcher2009} or observation of CO$_2$ clouds on the nightside \citep[p.~178]{Fletcher2009}.

\subsection{ASPERA--3 (\textit{Analyser of Space Plasmas and EneRgetic Atoms})}
ASPERA--3 is an instrument designed to study the interaction between solar wind and martian atmosphere. It comprises of four separate detectors. The first one is \textit{Neutral Particle Imager} (NPI) measuring the \textit{energetic neutral atom} (ENA) flux with high angular resolution. Another one neutral atoms sensor, the \textit{Neutral Particle Detector} (NPD), measures the neutral atom flux resolving energy and mass of the atoms. The other two instruments are aimed at electrically charged particles. The \textit{Electron Spectrometer} (ELS) is a top-hat electrostatic analyzer, while the \textit{Ion Mass Analyzer} (IMA) is an ion mass composition analyzer working with H$^+$, He$^{2+}$, He$^+$ and O$^+$ ions \citep[p.~122]{Chicarro2004}.

ASPERA--3 should focus on measuring ENAs in order to investigate the interaction between solar wind and martian atmosphere, to characterize the impact of plasma processes on atmospheric evolution and to obtain plasma and neutral gas distribution near Mars. It should also measure electrons and ions to complement ENA measurements, to study the dynamics and structure of plasma and to provide solar wind parameters \citep[p.~122]{Chicarro2004}.

To present some results of~ASPERA--3 we can mention discovering that the solar wind penetrates much deeper in martian atmosphere than was believed, being one of the atmospheric ions escape mechanisms \citep{Barabash2007}, detection of ENA jets caused by solar wind \citep[pp.~208--209]{Fletcher2009}, observing the ENA flux during Mars eclipse which laid foundation of a new method to measure planetary exosphere \citep[p.~209]{Fletcher2009} or proving there is a yet unidentified source of interplanetary ENAs \citep[pp.~209--212]{Fletcher2009}.

\subsection{MaRS (\textit{Mars Express Orbiter Radio Science})}
Opposite to the already described devices, the MaRS experiment doesn't have a dedicated physical device like a sensor or transmitter. Instead, it utilizes the communication antennas to perform radio occultation experiments. It can use either the parabolic \n[m]{1.6} diameter \textit{High Gain Antenna} or the smaller \textit{Low Gain Antennas}. The second part of the occultation experiments (namely the receivers) cannot be carried on board MEX, because they need to be on the opposite side of Mars than MEX is. Thus, the receivers are placed on Earth (Kourou, French Guayana; Darmstadt, Germany; Perth, Australia; plus 3 NASA's \textit{Deep Space Network} telescopes in Goldstone, USA; Madrid, Spain and Canberra, Australia). The experiment uses two frequency bands -- the S-band at \n[GHz]{2.1} and the X-band at \n[GHz]{7.1} \citep[pp.~153--154]{Chicarro2004}.   

MaRS is intended to sound the neutral atmosphere to derive vertical density, pressure and temperature profiles, to sound the ionosphere as well (in order to get electron density profiles), to determine the dielectric properties of the surface, to detect gravity anomalies and to sound the solar corona at extra occasions \citep[p.~141]{Chicarro2004}.

MaRS contributed towards improving existing atmospheric \textit{global circulation models} \citep[p.~227]{Fletcher2009}, towards the discovery of so~called ``meteor layer'' of atmosphere containing ionized metallic atoms brought into the atmosphere by meteoric impacts \citep[p.~230]{Fletcher2009} and towards refining the crustal structure \citep[p.~234]{Fletcher2009}.

\subsection{Beagle 2}
Beagle~2 is the lander module MEX was equipped with. It detached from the spacecraft on 19~December~2003 (6~days before MEX orbit entry) and its touchdown was planned to 25~December~2003. However, it hasn't transmitted any signal after the martian atmosphere entry. As of February~2004 it was declared lost. No particular reason came out on inquiry into its fault \citep{Bonacina2004}.

To accomplish its main goal (searching for existing or former life, or at least for conditions allowing development of life in the past) it was equipped with several scientific tools. To begin with, the \textit{Gas Analysis Package} is a mass spectrometer used for examining the surrounding atmospheric gases as well as rock and soil samples (heated in ovens in order to vaporize). The \textit{X-Ray Spectrometer} studies the composition of rock and soil samples using X-Ray fluorescence spectrometry being able to detect metals like Fe, Mg, Al, Ti and others. Another spectrometer, the \textit{M�ssbauer Spectrometer} is able to analyze materials containing iron. Its \textit{Stereo Camera System} was intended to acquire stereoscopic images of the landing site in various spectral ranges. One of the largest contributions to Beagle's main goal should have been brought by the \textit{Microscopic Imager} (by searching for microscopic fossils). As a support for all the mentioned systems, the \textit{Planetary Underground Tool} handles soil samples acquisition using a \n[m]{1.5}~long drill. There is also a grinder available for removing unwanted material from the samples or the surrounding surface. There are also several sensors attached to Beagle~2 - the \textit{oxidant sensor} monitoring the oxidizing effects of martian atmosphere, the \textit{UV sensor} capturing the UVA and UVB spectral ranges (which are lethal for organisms), the \textit{wind sensor} recording the speed and direction of wind, the \textit{air pressure sensor} with resolution \n[hPa]{0.003}, the \textit{air temperature sensor} with accuracy about \n[K]{0.01} and finally the \textit{dust impact monitor} measuring the magnitude and impact rate of dust particles \citep[pp.~165--191]{Chicarro2004}.

\begin{figure}
	\centering
	\includegraphics[width=140mm]{images/Beagle_2_lander.jpg}
	\caption{Visualization of the Beagle~2 lander on martian surface. Credit: Beagle~2 \citep{Beagle2}}
	\label{fig:beagle2}
\end{figure}

\section{The MARSIS experiment}
In this section we will discuss the individual parts of the MARSIS experiment. We are going to briefly describe the physical background of the experiments as well as the technical solution of the measurement mechanisms.

\subsection{Subsurface sounding}
The subsurface sounding attempts to detect the borders of the \textit{cryosphere}, which is the crust layer in which the temperature remains constantly under the water-freezing point. Such borders can be identified owing to different dielectric properties of liquid water and ice or ice and atmospheric gases. The deeper border can be a water--ice interface because the cryosphere ends where the internal planetary heat flow raises the temperature above the water-melting point (so if there is a liquid water reservoir under the cryosphere, it can be detected). This interface is expected to be at 0--\n[m]{5000} depth. On the other hand, the higher border can be formed by the desiccated megaregolith (martian soil) where the desiccation is caused by subsurface ice sublimation (estimated to be at depths between 0 and \n[m]{1000}) \citep[pp.~52--53]{Chicarro2004}.

As described in part \ref{ssec:marisIntro}, MARSIS can utilize a \n[m]{40} long dipole antenna as well as a \n[m]{7} monopole one. Only the dipole antenna is used for signal transmission (generating up to \n[W]{10} strong signal), and both antennas for signal receipt. It can sound using one of the four subsurface frequency bands centered at \n{1.8}, \n{3}, \n{4} and \n[MHz]{5}, every one having its bandwidth of \n[MHz]{1}. When MEX operates on the dayside of Mars, the ionosphere doesn't allow to use lower frequency bands for sounding (see section \ref{ssec:ionospericSounding}), so only the last two bands can be used. On the nightside, all four bands get through the ionosphere and allow to sound deeper subsurface. However, due to the limitations given by the MEX spacecraft, only echoes from depths up to 5--\n[km]{8} can be detected \citep[p.~57]{Chicarro2004}. 

The subsurface sounder mode is based on the fact that the radar waves reflect not only on the surface, but also on subsurface dielectric discontinuities. In addition, the velocity of the waves decreases proportionally to the material loss tangent, the wavelength and the depth -- which facilitates computing the depth of subsurface interfaces \citep[p.~56]{Chicarro2004}.  

\subsection{Surface sounding}
It arises from the previous paragraphs that the surface sounding mode is a ``subset'' of the subsurface sounding mode, taking only the ``topmost'' echoes into account. Therefore, no additional operation modes are present for just the surface sounding. 

The surface sounding is used to create a topography of the surface with lateral resolution 5--\n[km]{9}. This topography further serves for improving the accuracy of statistical topography models which describe the surface in the means of a random distribution of heights \citep[p.~54]{Chicarro2004}.

\subsection{Ionospheric sounding}
\label{ssec:ionospericSounding}
The basic reason for studying the ionosphere is that it stops propagation of electromagnetic waves with frequencies below the local \textit{electron plasma frequency} $f_p = 8980 \sqrt{N_e}\mathrm{\ Hz}$, where $N_e$ is the local electron density in cm$^{-3}$. All vertical waves with frequencies below the maximum electron plasma frequency, $f_p$(max), are reflected back at a place with the same frequency as the waves have. This maximum is usually located at the heights 125--\n[km]{150} and amounts up to \n[MHz]{4} on the dayside and \n[kHz]{800} on the nightside \citep[pp.~55--56]{Chicarro2004}.

MARSIS uses two methods -- a passive and an active one. The passive method measures thermal emission at the local electron plasma frequency. The active method -- the one of our interest -- sounds the ionosphere with the radar in 160~frequency steps ranging from \n[kHZ]{100} to \n[MHz]{5.4}. Every pulse has a duration of \n[ms]{91.4}. With such a sampling it is possible to construct vertical profiles of the electron plasma frequency (and also electron density). Besides the normal ionospheric sounding mode, MARSIS also provides a special interleaved mode switching periodically between the subsurface sounding and ionosphere sounding modes. This yields a method to remove the ionospheric effects from the subsurface sounding results \citep[p.~58]{Chicarro2004}.

Adding to the ionospheric and surface echoes, there are three more (unexpected \citep[p.~1930]{Gurnett2005}, but useful) signal patterns detectable using the ionospheric sounding. Namely, oblique ionospheric echoes, electron plasma oscillation harmonics and electron cyclotron echoes. We will describe all of them in the following sections after presenting the concept of ionograms.

\section{Ionograms}

\begin{figure}
	\centering
	\includegraphics[width=140mm]{images/ionogram_example.jpg}
	\caption{Example of a ionogram showing most of the detectable features like ionospheric echo, surface reflection, electron cyclotron echoes and electron plasma oscillation harmonics. No oblique ionospheric echo is present. The vertical axis shows delay time in ms, the horizontal axis stands for frequency in MHz and color codes the spectral density of the received electric field in V$^2$m$^{-2}$Hz$^{-1}$. Based on real data obtained from \citep{FTP}.}
	\label{fig:example_ionogram}
\end{figure}

Ionograms are the basic visualization of the ionospheric sounding data. Akalin \citep{Akalin2010} defines ionograms in the following precise way:
\begin{quote} 
Ionograms are produced by transmitting a short pulse at a fixed frequency, $f$, and measuring the received intensity at 80~consecutive values of the time delay, ${\Updelta}t$, spaced \n[{\upmu}s]{91.4} apart. The frequency is then incremented and the process is repeated. For each of 160~frequencies, quasi-logarithmically spaced between \n{0.1} and \n[MHz]{5.5}, there are 80~delay time bins, spaced \n[{\upmu}s]{91.4} apart, beginning \n[{\upmu}s]{162.5} after the end of the sounding pulse. Ionograms represent received intensity as a function of time delay and frequency. As shown by the ionogram in Fig. \ref{fig:example_ionogram}, time delay is displayed in milliseconds along the vertical axis, frequency is displayed in megahertz along the horizontal axis, and the color bar represents the received electric field spectral density in V$^2$m$^{-2}$Hz$^{-1}$.
\end{quote}

Several more or less continuous patterns can be found in the example ionogram. Some of them form repetitious patterns. It can be also seen that the data are very noisy. The example ionogram is rather rare, because often just one or two such patterns occur in a single ionogram. There are also ionograms consisting entirely of noise. The subsequent sections will discuss all the patterns and their physical meaning.

\subsection{Ionospheric echo}
As seen in Fig. \ref{fig:example_ionogram}, the ionospheric echo is a horizontally oriented non-straight line. It usually appears in the lower half of the ionogram (delay times about 4 to \n[ms]{5}). Its left end is located where the local f$_p$ frequency starts to be higher than the sounding frequency, which is most often somewhere below \n[MHz]{1}. Its right end should be placed at f$_p$(max) frequency, where all higher-frequency waves pass to the surface \citep[p.~1929]{Gurnett2005}. 

There is often a sharp cusp at the right end of the echo. ``The cusp occurs because the propagation speed of the wave packet (i.e., the group velocity) is very small over an increasingly long path length as the wave frequency approaches f$_p$(max)'' \citep[p.~1929]{Gurnett2005}. On the other hand, the echo often doesn't extend up to f$_p$(max) \citep[p.~1930]{Gurnett2005}.

As we have mentioned earlier, it is possible to read out the local electron plasma frequency from the echo, thus obtaining the electron density vertical profile. In order to extract the profile, it is needed to identify the curve fitting the echo. Automatic identification of such curve is one of the goals of this work. Especially correct estimation of the right end would be helpful if the cusp is present.  

\subsection{Surface echo}

\subsection{Oblique ionospheric echo}
\label{ssec:oblique}

\subsection{Electron cyclotron echoes}
\label{ssec:cyclotronEchoes}

\subsection{Electron plasma oscillation harmonics}